% inspired from http://tex.stackexchange.com/questions/57152/how-to-draw-graphs-in-latex
\documentclass{article}
\thispagestyle{empty}

\usepackage[francais]{babel}
\usepackage[utf8]{inputenc}

\usepackage{bm}
\usepackage{tikz}
\usetikzlibrary{positioning}
\usetikzlibrary{arrows}
\usepackage{geometry}
\usepackage{diagbox}

\geometry{hmargin=1cm, vmargin=1cm}

\begin{document}

% Exemple de cycle améliorant

\begin{tikzpicture}[ shorten >=1pt, ->, >=stealth, auto, node distance = 3cm, thick, 
   node/.style = {circle, draw,
                     font = \sffamily\Large\bfseries}]
   \node[node, fill = blue!20] (1) {$1$};
   \node[node] (2) [below right = 1.5cm and 3cm of 1]{$2$};
   \node[node] (3) [above right = 1.5cm and 3cm of 1]{$3$};
   \node[node, fill=green!20] (4) [below right = 1.5cm and 3cm of 3]{$4$};

   \path[color = red, ultra thick] (1) edge node {$-2$} (2);
   \path[color = red, ultra thick] (2) edge node {$1$} (3);
   \path (2) edge node {$3$} (4);
   \path[color = red, ultra thick] (3) edge node {$-2$} (1);
   \path (3) edge node {$-4$} (4);
\end{tikzpicture}

% Exemple de graphe pour le pseudo-code itératif

\vspace{1cm}

\begin{tikzpicture}[ shorten >=1pt, ->, >=stealth, auto, node distance = 3cm, thick, 
   node/.style = {circle, draw,
                     font = \sffamily\Large\bfseries}]
   \node[node, fill = blue!20] (1) {$1$};
   \node[node] (2) [below right = 1.3cm and 2cm of 1]{$2$};
   \node[node] (3) [above right = 1cm and 3cm of 1]{$3$};
   \node[node] (4) [above right = 0.1cm and 2.5cm of 2]{$4$};
   \node[node, fill=green!20] (5) [above right = 0.5cm and 2.5cm of 4]{$5$};

   \path (1) edge node {$2$} (2);
   \path (1) edge node {$-1$} (3);
   \path (2) edge node {$3$} (3);
   \path (2) edge node {$-2$} (4);
   \path (3) edge node {$2$} (4);
   \path (3) edge node {$2$} (5);
   \path (4) edge node {$-4$} (5);
\end{tikzpicture}

% Exemple init

\newpage

\begin{tikzpicture}[ scale=0.8, every node/.style={transform shape},
   shorten >=1pt, ->, >=stealth, auto, node distance = 3cm, thick, 
   node/.style = {circle, draw,
                     font = \sffamily\Large\bfseries}]
   \node[node, fill = blue!20] (1) {$1$};
   \node[node] (2) [below right = 1.3cm and 2cm of 1]{$2$};
   \node[node] (3) [above right = 1cm and 3cm of 1]{$3$};
   \node[node] (4) [above right = 0.1cm and 2.5cm of 2]{$4$};
   \node[node, fill=green!20] (5) [above right = 0.5cm and 2.5cm of 4]{$5$};

   \path (1) edge node {$2$} (2);
   \path (1) edge node {$-1$} (3);
   \path (2) edge node {$3$} (3);
   \path (2) edge node {$-2$} (4);
   \path (3) edge node {$2$} (4);
   \path (3) edge node {$2$} (5);
   \path (4) edge node {$-4$} (5);
\end{tikzpicture}

\begin{table}[h]
   \begin{tabular}{|c|c|c|c|c|c|}
      \hline
      \diagbox{Etape}{Nœud} & 1 & 2 & 3 & 4 & 5 \\ \hline
      \textbf{0} & $\bm{+\infty}$ & $\bm{+\infty}$ & $\bm{+\infty}$ & $\bm{+\infty}$ & \textbf{0} \\ \hline
      1 & $+\infty$ & $+\infty$ & $+\infty$ & $+\infty$ & $+\infty$ \\ \hline
      2 & $+\infty$ & $+\infty$ & $+\infty$ & $+\infty$ & $+\infty$ \\ \hline
      3 & $+\infty$ & $+\infty$ & $+\infty$ & $+\infty$ & $+\infty$ \\ \hline
      4 & $+\infty$ & $+\infty$ & $+\infty$ & $+\infty$ & $+\infty$ \\ \hline
   \end{tabular}
\end{table}

% Exemple tour 0

\begin{tikzpicture}[ scale=0.8, every node/.style={transform shape},
   shorten >=1pt, ->, >=stealth, auto, node distance = 3cm, thick, 
   node/.style = {circle, draw,
                     font = \sffamily\Large\bfseries}]
   \node[node, fill = blue!20] (1) {$1$};
   \node[node] (2) [below right = 1.3cm and 2cm of 1]{$2$};
   \node[node] (3) [above right = 1cm and 3cm of 1]{$3$};
   \node[node] (4) [above right = 0.1cm and 2.5cm of 2]{$4$};
   \node[node, fill=green!20] (5) [above right = 0.5cm and 2.5cm of 4]{$5$};

   \path (1) edge node {$2$} (2);
   \path (1) edge node {$-1$} (3);
   \path (2) edge node {$3$} (3);
   \path (2) edge node {$-2$} (4);
   \path (3) edge node {$2$} (4);
   \path (3) edge node {$2$} (5);
   \path (4) edge node {$-4$} (5);
\end{tikzpicture}

\begin{table}[h]
   \begin{tabular}{|c|c|c|c|c|c|}
      \hline
      \diagbox{Etape}{Nœud} & 1 & 2 & 3 & 4 & 5 \\ \hline
      \textbf{0} & $\bm{+\infty}$ & $\bm{+\infty}$ & \textbf{2} & \textbf{-4} & \textbf{0} \\ \hline
      1 & $+\infty$ & $+\infty$ & $+\infty$ & $+\infty$ & $+\infty$ \\ \hline
      2 & $+\infty$ & $+\infty$ & $+\infty$ & $+\infty$ & $+\infty$ \\ \hline
      3 & $+\infty$ & $+\infty$ & $+\infty$ & $+\infty$ & $+\infty$ \\ \hline
      4 & $+\infty$ & $+\infty$ & $+\infty$ & $+\infty$ & $+\infty$ \\ \hline
   \end{tabular}
\end{table}

\end{document}
