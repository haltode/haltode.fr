% inspired from http://tex.stackexchange.com/questions/63612/tikz-tree-drawing-with-comments-to-each-level and http://tex.stackexchange.com/questions/45808/tikz-grid-lines
% https://tex.stackexchange.com/questions/64148/tikz-label-on-tree-edge
% https://tex.stackexchange.com/questions/329451/tikz-forest-drawing-upon-a-subtree
% https://tex.stackexchange.com/questions/249694/how-can-i-increase-the-padding-around-nodes-in-a-fitted-box
% https://tex.stackexchange.com/questions/294245/nice-empty-nodes-broken-by-forest-2-0
% https://tex.stackexchange.com/questions/274219/drawing-a-multicolored-grid-using-tikz
\documentclass{article}
\usepackage[francais]{babel}
\usepackage[utf8]{inputenc}
\usepackage[T1]{fontenc}
\usepackage[margin=0.25in]{geometry}

\usepackage{tikz}
\usepackage{forest}
\usepackage{pgfplots}

\usetikzlibrary{positioning, fit, calc}

\begin{document}

% Possibilité d'arrangement des objets

% \forestset{nice empty nodes/.style={
% delay={where content={}{shape=coordinate,for parent={for children={child anchor=north}}}{}} }}

\forestset{nice empty nodes/.style={
   delay={where content={}{shape=coordinate,for siblings={anchor=north}}{}}
}}

\begin{forest}
for tree={circle,draw, l sep=20pt, s sep=20pt}
[,nice empty nodes
   [1, fill=red!20
      [2, fill=red!20
         [3, fill=red!20]
         [3, fill=green!20]
      ]
      [2, fill=green!20
         [3, fill=red!20]
         [3, fill=green!20]
      ]
   ]
   [1, fill=green!20
      [2, fill=red!20
         [3, fill=red!20]
         [3, fill=green!20]
      ]
      [2, fill=green!20
         [3, fill=red!20]
         [3, fill=green!20]
      ]
   ]
]
\end{forest}

% Limite de l'algorithme naïf : répétition inutile d'appels récursifs
\vspace{1cm}
\begin{forest}
for tree={scale=0.5}
[${I(0,150)}$
   [${I(1,150)}$
      [${I(2,150)}$
         [${I(3,150)}$
            [${I(4,150)}$
            ]
            [${I(4,125)}$
            ]
         ]
         [${I(3,140)}$
            [${I(4,140)}$
            ]
            [${I(4,115)}$
            ]
         ]
      ]
      [${I(2,130)}$, name=top1
         [${I(3,130)}$
            [${I(4,130)}$, name=bottom_left1
            ]
            [${I(4,105)}$
            ]
         ]
         [${I(3,120)}$
            [${I(4,120)}$
            ]
            [${I(4,95)}$, name=bottom_right1
            ]
         ]
      ]
   ]
   [${I(1,130)}$
      [${I(2,130)}$, name=top2
         [${I(3,130)}$
            [${I(4,130)}$, name=bottom_left2
            ]
            [${I(4,105)}$
            ]
         ]
         [${I(3,120)}$
            [${I(4,120)}$
            ]
            [${I(4,95)}$, name=bottom_right2
            ]
         ]
      ]
      [${I(2,110)}$
         [${I(3,110)}$
            [${I(4,110)}$
            ]
            [${I(4,85)}$
            ]
         ]
         [${I(3,100)}$
            [${I(4,100)}$
            ]
            [${I(4,75)}$
            ]
         ]
      ]
   ]
]
\node[inner sep=0.04cm, draw=blue!20, thick, fit=(top1)(bottom_left1)(bottom_right1)] {};
\node[inner sep=0.04cm, draw=blue!20, thick, fit=(top2)(bottom_left2)(bottom_right2)] {};
\end{forest}

% 

\vspace{2cm}

\begin{tikzpicture}[every node/.style={minimum size=.5cm-\pgflinewidth, outer sep=0pt}]

   \draw[step=.5cm,color=gray!50] (-1,-1) grid (2,2);
   \draw[thick,->] (-1,2) -- (-1,-1);
   \draw[thick,->] (-1,2) -- (2,2);
   \node[] at (-1.2,2.2) {\footnotesize 0};
   \node[] at (-1.2,-1.3) {\footnotesize nb objets};
   \node[] at (2.2,2.3) {\footnotesize poids};

   % Cas de bases
   \node[circle,fill=green!60,scale=0.3] at (-0.75,+1.75) {};
   \node[circle,fill=green!60,scale=0.3] at (-0.25,+1.75) {};
   \node[circle,fill=green!60,scale=0.3] at (+0.25,+1.75) {};
   \node[circle,fill=green!60,scale=0.3] at (+0.75,+1.75) {};
   \node[circle,fill=green!60,scale=0.3] at (+1.25,+1.75) {};
   \node[circle,fill=green!60,scale=0.3] at (+1.75,+1.75) {};

   % Dépendance des calculs
   \node[circle,fill=blue!60,scale=0.3] at (+1.25,+0.25) {};
   \draw[thick,->,blue!60] (1.25,0.25) -- (1.25,0.75);
   \draw[thick,->,blue!60] (1.25,0.25) -- (0.25,0.75);
   \node[text=blue!80] at (0.15,0.9) {\tiny -objet.poids};

   % Problème final
   \node[circle,fill=red!60,scale=0.3] at (+1.75,-0.75) {};

\end{tikzpicture}

\end{document}
